% Options for packages loaded elsewhere
\PassOptionsToPackage{unicode}{hyperref}
\PassOptionsToPackage{hyphens}{url}
%
\documentclass[
]{article}
\usepackage{amsmath,amssymb}
\usepackage{lmodern}
\usepackage{iftex}
\ifPDFTeX
  \usepackage[T1]{fontenc}
  \usepackage[utf8]{inputenc}
  \usepackage{textcomp} % provide euro and other symbols
\else % if luatex or xetex
  \usepackage{unicode-math}
  \defaultfontfeatures{Scale=MatchLowercase}
  \defaultfontfeatures[\rmfamily]{Ligatures=TeX,Scale=1}
\fi
% Use upquote if available, for straight quotes in verbatim environments
\IfFileExists{upquote.sty}{\usepackage{upquote}}{}
\IfFileExists{microtype.sty}{% use microtype if available
  \usepackage[]{microtype}
  \UseMicrotypeSet[protrusion]{basicmath} % disable protrusion for tt fonts
}{}
\makeatletter
\@ifundefined{KOMAClassName}{% if non-KOMA class
  \IfFileExists{parskip.sty}{%
    \usepackage{parskip}
  }{% else
    \setlength{\parindent}{0pt}
    \setlength{\parskip}{6pt plus 2pt minus 1pt}}
}{% if KOMA class
  \KOMAoptions{parskip=half}}
\makeatother
\usepackage{xcolor}
\usepackage[margin=1in]{geometry}
\usepackage{color}
\usepackage{fancyvrb}
\newcommand{\VerbBar}{|}
\newcommand{\VERB}{\Verb[commandchars=\\\{\}]}
\DefineVerbatimEnvironment{Highlighting}{Verbatim}{commandchars=\\\{\}}
% Add ',fontsize=\small' for more characters per line
\usepackage{framed}
\definecolor{shadecolor}{RGB}{248,248,248}
\newenvironment{Shaded}{\begin{snugshade}}{\end{snugshade}}
\newcommand{\AlertTok}[1]{\textcolor[rgb]{0.94,0.16,0.16}{#1}}
\newcommand{\AnnotationTok}[1]{\textcolor[rgb]{0.56,0.35,0.01}{\textbf{\textit{#1}}}}
\newcommand{\AttributeTok}[1]{\textcolor[rgb]{0.77,0.63,0.00}{#1}}
\newcommand{\BaseNTok}[1]{\textcolor[rgb]{0.00,0.00,0.81}{#1}}
\newcommand{\BuiltInTok}[1]{#1}
\newcommand{\CharTok}[1]{\textcolor[rgb]{0.31,0.60,0.02}{#1}}
\newcommand{\CommentTok}[1]{\textcolor[rgb]{0.56,0.35,0.01}{\textit{#1}}}
\newcommand{\CommentVarTok}[1]{\textcolor[rgb]{0.56,0.35,0.01}{\textbf{\textit{#1}}}}
\newcommand{\ConstantTok}[1]{\textcolor[rgb]{0.00,0.00,0.00}{#1}}
\newcommand{\ControlFlowTok}[1]{\textcolor[rgb]{0.13,0.29,0.53}{\textbf{#1}}}
\newcommand{\DataTypeTok}[1]{\textcolor[rgb]{0.13,0.29,0.53}{#1}}
\newcommand{\DecValTok}[1]{\textcolor[rgb]{0.00,0.00,0.81}{#1}}
\newcommand{\DocumentationTok}[1]{\textcolor[rgb]{0.56,0.35,0.01}{\textbf{\textit{#1}}}}
\newcommand{\ErrorTok}[1]{\textcolor[rgb]{0.64,0.00,0.00}{\textbf{#1}}}
\newcommand{\ExtensionTok}[1]{#1}
\newcommand{\FloatTok}[1]{\textcolor[rgb]{0.00,0.00,0.81}{#1}}
\newcommand{\FunctionTok}[1]{\textcolor[rgb]{0.00,0.00,0.00}{#1}}
\newcommand{\ImportTok}[1]{#1}
\newcommand{\InformationTok}[1]{\textcolor[rgb]{0.56,0.35,0.01}{\textbf{\textit{#1}}}}
\newcommand{\KeywordTok}[1]{\textcolor[rgb]{0.13,0.29,0.53}{\textbf{#1}}}
\newcommand{\NormalTok}[1]{#1}
\newcommand{\OperatorTok}[1]{\textcolor[rgb]{0.81,0.36,0.00}{\textbf{#1}}}
\newcommand{\OtherTok}[1]{\textcolor[rgb]{0.56,0.35,0.01}{#1}}
\newcommand{\PreprocessorTok}[1]{\textcolor[rgb]{0.56,0.35,0.01}{\textit{#1}}}
\newcommand{\RegionMarkerTok}[1]{#1}
\newcommand{\SpecialCharTok}[1]{\textcolor[rgb]{0.00,0.00,0.00}{#1}}
\newcommand{\SpecialStringTok}[1]{\textcolor[rgb]{0.31,0.60,0.02}{#1}}
\newcommand{\StringTok}[1]{\textcolor[rgb]{0.31,0.60,0.02}{#1}}
\newcommand{\VariableTok}[1]{\textcolor[rgb]{0.00,0.00,0.00}{#1}}
\newcommand{\VerbatimStringTok}[1]{\textcolor[rgb]{0.31,0.60,0.02}{#1}}
\newcommand{\WarningTok}[1]{\textcolor[rgb]{0.56,0.35,0.01}{\textbf{\textit{#1}}}}
\usepackage{graphicx}
\makeatletter
\def\maxwidth{\ifdim\Gin@nat@width>\linewidth\linewidth\else\Gin@nat@width\fi}
\def\maxheight{\ifdim\Gin@nat@height>\textheight\textheight\else\Gin@nat@height\fi}
\makeatother
% Scale images if necessary, so that they will not overflow the page
% margins by default, and it is still possible to overwrite the defaults
% using explicit options in \includegraphics[width, height, ...]{}
\setkeys{Gin}{width=\maxwidth,height=\maxheight,keepaspectratio}
% Set default figure placement to htbp
\makeatletter
\def\fps@figure{htbp}
\makeatother
\setlength{\emergencystretch}{3em} % prevent overfull lines
\providecommand{\tightlist}{%
  \setlength{\itemsep}{0pt}\setlength{\parskip}{0pt}}
\setcounter{secnumdepth}{-\maxdimen} % remove section numbering
\usepackage{bbm}
\ifLuaTeX
  \usepackage{selnolig}  % disable illegal ligatures
\fi
\IfFileExists{bookmark.sty}{\usepackage{bookmark}}{\usepackage{hyperref}}
\IfFileExists{xurl.sty}{\usepackage{xurl}}{} % add URL line breaks if available
\urlstyle{same} % disable monospaced font for URLs
\hypersetup{
  pdftitle={A Collection of Methods to Estimate Parameters of Different Tempered Stable Distributions},
  pdfauthor={Cedric Juessen},
  hidelinks,
  pdfcreator={LaTeX via pandoc}}

\title{A Collection of Methods to Estimate Parameters of Different
Tempered Stable Distributions}
\author{Cedric Juessen}
\date{October 31rd, 2022}

\begin{document}
\maketitle

\hypertarget{tempstable}{%
\section{\texorpdfstring{TempStable }{TempStable }}\label{tempstable}}

A collection of methods to estimate parameters of different tempered
stable distributions. Currently, there are three different tempered
stable distributions to choose from: Tempered stable subordinator
distribution, classical tempered stable distribution, normal tempered
stable distribution. The package also provides functions to compute
characteristic functions and tools to run Monte Carlo simulations.

The main function of this package are briefly described below:

\begin{itemize}
\tightlist
\item
  Main function: TemperedEstim() computes all the information about the
  estimator. It allows the user to choose the preferred method and
  several related options.
\item
  Characteristic function, density function, probability function and
  other functions for every tempered stable distribution mentioned
  above. E.g. charSTS(), dCTS(), \ldots{}
\item
  Monte Carlo simulation: a tool to run a Monte Carlo simulation
  (TemperedEstim\_Simulation()) is provided and can save output files or
  produce statistical summary.
\end{itemize}

The package was developed by Till Massing and Cedric Jüssen and is
structurally based on the ``StableEstim'' package by Tarak Kharrat and
Georgi N. Boshnakov.

\hypertarget{installation}{%
\subsection{Installation}\label{installation}}

You can install the development version of TempStable from
\href{https://github.com/}{GitHub} with:

\begin{Shaded}
\begin{Highlighting}[]
\CommentTok{\# install.packages("devtools")}
\NormalTok{devtools}\SpecialCharTok{::}\FunctionTok{install\_github}\NormalTok{(}\StringTok{"cedricjuessen/TempStable"}\NormalTok{)}
\end{Highlighting}
\end{Shaded}

\hypertarget{example}{%
\subsection{Example}\label{example}}

This is a basic example which shows you how to solve a common problem:

\begin{Shaded}
\begin{Highlighting}[]
\FunctionTok{library}\NormalTok{(TempStable)}
\DocumentationTok{\#\# basic example code}
\CommentTok{\# Such a simulation can take a very long time. Therefore, it can make sense to }
\CommentTok{\# parallelise after Monte Carlo runs. Parallelisation of the simulation is not }
\CommentTok{\# yet part of the package. }

\CommentTok{\# For testing purposes, the amount of runs and parameters is greatly reduced. }
\CommentTok{\# Therefore, the result is not meaningful. To start a meaningful simulation, the}
\CommentTok{\# SampleSize could be, for example, 1000 and MCParam also 1000.}
\NormalTok{thetaT }\OtherTok{\textless{}{-}} \FunctionTok{c}\NormalTok{(}\FloatTok{1.5}\NormalTok{,}\DecValTok{1}\NormalTok{,}\DecValTok{1}\NormalTok{,}\DecValTok{1}\NormalTok{,}\DecValTok{1}\NormalTok{,}\DecValTok{0}\NormalTok{)}
\NormalTok{res\_CTS\_ML\_size10 }\OtherTok{\textless{}{-}} \FunctionTok{TemperedEstim\_Simulation}\NormalTok{(}\AttributeTok{ParameterMatrix =} \FunctionTok{rbind}\NormalTok{(thetaT),}
                                               \AttributeTok{SampleSizes =} \FunctionTok{c}\NormalTok{(}\DecValTok{10}\NormalTok{), }\AttributeTok{MCparam =} \DecValTok{10}\NormalTok{,}
                                               \AttributeTok{TemperedType =} \StringTok{"Classic"}\NormalTok{, }\AttributeTok{Estimfct =} \StringTok{"ML"}\NormalTok{,}
                                               \AttributeTok{saveOutput =} \ConstantTok{FALSE}\NormalTok{)}
\CommentTok{\#\textgreater{} {-}{-}{-}{-}{-}{-}{-}{-}{-}{-}{-}{-}{-}{-}{-}{-} Alpha=1.5 *** DeltaP=1 *** DeltaM=1 *** LambdaP=1 *** LambdaM=1 *** mu=0 {-}{-}{-}{-}{-}{-}{-}{-}{-}{-}{-}{-}{-}{-}{-} }
\CommentTok{\#\textgreater{} *** Iter 1/10 *** Estimated Remaining Time: 0h2min2sec. *** }
\CommentTok{\#\textgreater{} *** Iter 2/10 *** Estimated Remaining Time: 0h2min1sec. *** }
\CommentTok{\#\textgreater{} *** Iter 3/10 *** Estimated Remaining Time: 0h4min3sec. *** }
\CommentTok{\#\textgreater{} *** Iter 4/10 *** Estimated Remaining Time: 0h3min4sec. ***}
\CommentTok{\#\textgreater{} Warning in log(densis): NaNs wurden erzeugt}

\CommentTok{\#\textgreater{} Warning in log(densis): NaNs wurden erzeugt}
\CommentTok{\#\textgreater{} *** Iter 5/10 *** Estimated Remaining Time: 0h0min52sec. ***}
\CommentTok{\#\textgreater{} Warning in log(densis): NaNs wurden erzeugt}
\CommentTok{\#\textgreater{} *** Iter 6/10 *** Estimated Remaining Time: 0h0min30sec. *** }
\CommentTok{\#\textgreater{} *** Iter 7/10 *** Estimated Remaining Time: 0h0min51sec. ***}
\CommentTok{\#\textgreater{} Warning in log(densis): NaNs wurden erzeugt}
\CommentTok{\#\textgreater{} *** Iter 8/10 *** Estimated Remaining Time: 0h0min15sec. ***}
\CommentTok{\#\textgreater{} Warning in log(densis): NaNs wurden erzeugt}
\CommentTok{\#\textgreater{} *** Iter 9/10 *** Estimated Remaining Time: 0h0min7sec. ***}
\CommentTok{\#\textgreater{} Warning in log(densis): NaNs wurden erzeugt}
\CommentTok{\#\textgreater{} *** Iter 10/10 *** Estimated Remaining Time: 0h0min0sec. ***}

\FunctionTok{colMeans}\NormalTok{(}\FunctionTok{sweep}\NormalTok{(res\_CTS\_ML\_size10}\SpecialCharTok{$}\NormalTok{outputMat[,}\DecValTok{9}\SpecialCharTok{:}\DecValTok{14}\NormalTok{],}\DecValTok{2}\NormalTok{,thetaT), }\AttributeTok{na.rm =} \ConstantTok{TRUE}\NormalTok{)}
\CommentTok{\#\textgreater{}     alphaE    delta+E    delta{-}E   lambda+E   lambda{-}E        muE }
\CommentTok{\#\textgreater{} {-}1.3042931  2.6728708  3.5903234  1.2565026  1.5549668  0.4091464}
\end{Highlighting}
\end{Shaded}


\end{document}
